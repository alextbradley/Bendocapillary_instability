% This is file JFM2esam.tex
% first release v1.0, 20th October 1996
%       release v1.01, 29th October 1996
%       release v1.1, 25th June 1997
%       release v2.0, 27th July 2004
%       release v3.0, 16th July 2014
%   (based on JFMsampl.tex v1.3 for LaTeX2.09)
% Copyright (C) 1996, 1997, 2014 Cambridge University Press

\documentclass{jfm}
\usepackage{graphicx}
\usepackage{epstopdf, epsfig}
\usepackage{graphicx}
%\usepackage{epstopdf,epsfig}
\usepackage{newtxtext}
\usepackage{newtxmath}
\usepackage{hyperref}
\usepackage{color}
\usepackage{siunitx}

\newtheorem{lemma}{Lemma}
\newtheorem{corollary}{Corollary}


\newcommand{\dd}[2]{\frac{\mathrm{d} #1}{\mathrm{d} #2}}
\newcommand{\ddp}[2]{\frac{\partial #1}{\partial #2}}
\newcommand{\order}[1]{\mathcal{O}\left(#1\right)}
\newcommand{\red}[1]{{\color{red} #1}}
\newcommand{\blue}[1]{{\color{blue} #1}}
\newcommand{\cyan}[1]{{\color{cyan} #1}}
\renewcommand{\Pi}{P}
\renewcommand{\Lambda}{H} %change definition of perturbed quantities

\newcommand{\poisson}{\eta} %Poisson's ratio of the channel walls
\newcommand{\aspect}{a} %aspect ratio: \aspect = [z]/[y]
\newcommand{\jump}[1]{_{#1^{\pm}}} %jump conditions across interface
\newcommand{\amplitude}{\delta} %amplitude of perturbation in scaling
\newcommand{\bendability}{\nu}
%if you want to change variables for any reason?
\newcommand{\h}{h}
\newcommand{\x}{x}
\newcommand{\y}{y}
\newcommand{\that}{t}
\newcommand{\nablahat}{\nabla}

\newcommand\abceqn[2]{\refstepcounter{equation}
     \[
     \label{#1}
     #2
     \eqno{\text{(\theequation)}\text{a,b,c}}
     \]
}
\newcommand\abeqn[2]{\refstepcounter{equation}
     \[
     \label{#1}
     #2
     \eqno{\text{(\theequation)}\text{a,b}}
     \]
}


\newcommand\abcdeqn[2]{\refstepcounter{equation}
     \[
     \label{#1}
     #2
     \eqno{\text{(\theequation)}\text{a,b,c,d}}
     \]
}

\shorttitle{Modelling a Novel Bendocapillary Instability} %Elastically mediated 
\shortauthor{A. T. Bradley, I. J. Hewitt, D. Vella}

\title{Modelling a Novel Bendocapillary Instability.}

%\author{Alexander T. Bradley\aff{1}
%  Ian. J. Hewitt\aff{1}
% \and Dominic Vella\aff{1}\corresp{\email{vella@maths.ox.ac.uk}}}
 
 

\author{Alexander T. Bradley\aff{1},
  Ian J. Hewitt\aff{1},
 \and Dominic Vella\aff{1} \corresp{\email{vella@maths.ox.ac.uk}}}


 
\affiliation{\aff{1} Mathematical Institute, University of Oxford, Woodstock Rd, Oxford, OX2 6GG, United Kingdom} 

\begin{document}

\maketitle

\begin{abstract}
We develop and analyze a mathematical model of a system, consisting of a channel with two flexible walls and a rigid base part filled with liquid, which is susceptible to a novel bendocapillary instability. Inspiration for studying this system comes from experiments in which a `weaving' pattern emerges as droplets of liquid are condensed slowly into deformable microchannels. We describe equilibria of the system, and then use a combination of numerical methods, and asymptotic analysis in the limit of small channel wall deflections, to elucidate the key features of this instability: (1) configurations are always unstable to perturbations of sufficiently small wavenumber, (2) growth rates are highly sensitive to the volume of liquid in the channel, and (3) both wetting and non-wetting configurations are theoretically susceptible to the instability in the same channel. Insight into novel interfacial instabilities opens the possibility for their control and thus exploitation in processes such as microfabrication.
\end{abstract}
%part 1: demonstrating the instability with constant volume, for wetting and non-wetting. Model + equilibria + scaling argument(?) + numerics for bvp + asymptotics for both wetting and non-wetting cases. Think we want to show experiments here as further motivation, then say here we consider only the statics (i.e. what happens if I inject liquid very slowly)

%part 2: how does a time dependent volume change the picture? Brief model + describe base states + linear stability analysis + might need to do some non-linear analysis?

\graphicspath{{./figures/}}

\section{Introduction}\label{S:Introduction}
%Interfacial Instabilities are very common [some examples]
Interfacial instabilities play an important role in many familiar phenomena, such as the break up of a water column leaving a faucet~\citep{Plateau1873, Rayleigh1879PRSL}, the formation of trains of soap bubbles~\citep{Eggers2008RepProgPhys}, and the dendritic morphology of snowflakes~\citep{Langer1980RevModPhys}. The traditional motivation for understanding such instabilities is to prevent the problems that they cause, such as the early failure in zinc alkaline batteries~\citep{Gallaway2010Electrochem}, the collapse of foetal airways~\citep{Halpern1992JFM}, and reduced pattern quality in inkjet printing~\citep{Calvert2007Science}. Recently, however, attention has shifted towards understanding interfacial instabilities in order to utilize them; for example, interfacial instabilities have been successfully exploited to increase the packing density of carbon nanotube forests~\citep{Chakrapani2004PNAS}, manufacture periodic assemblies~\cite[e.g.][]{DeVolder2013Angewandte}, and trap colloids at surfaces~\citep{Pokroy2009Science}.

%these instabilites are usually unhelpful, but can be exploited? [I don't like including this sentence but applications important, and sets up why we want to understand these instabilities more -- understanding means we know conditions to suppress or exploit]

%In many of these examples, small scale fluid flow is important. On these scales, surface tension plays an important role.  Capillary flows arise from deformation of an interface, so channel shapes (morphological effects) are very important
%we know that geometry and wettability are very important. E.g. the Saffman-Taylor instability suppressed by changing the angle of the plates[Al-Housseiny] and evaporative instabilities that arise in drying MEMS might be suppressed by changing the wettability of the liquid [Ledesma-Aguilar].
Small scale fluid flow that is confined by solid boundaries is fundamental to many situations in which interfacial instabilities are encountered.  On such scales, surface forces dominate over body forces, and thus capillarity plays a dominant role in controlling the flow. Since capillary flows arise from the deformation of an interface, the shape of the confinement is critical to the behaviour of such instabilities. For example, the Saffman-Taylor (or `viscous fingering') instability~\citep{Saffman1958PRSL} that classically occurs when a liquid of higher viscosity displaces a liquid of lower viscosity in channel with a uniform width can be suppressed by varying either the channel width in space~\citep{AlHousseiny2012NaturePhysics, AlHousseiny2013PhysFlu, Reyssat2014JFM} or in time~\citep{Zheng2015PRL}. Indeed, such tapering can even promote an `opposite' instability in which less viscous liquid is able to displace more viscous liquid from the apex of a wedge~\citep{Keiser2016JFM}. On scales that are smaller still, evaporation driven instabilities that are common in drying of microelectromechanical systems are known to be to have a sensitive dependence on the geometry of the channel~\citep{LedesmaAguilar2017SoftMatter, Hadjittofis2016JFM}.
%When the walls are flexible, new possibilities open up. For example, the Saffman Taylor instability can be suppressed by replacing one of the walls by a flexible walls 

\begin{figure}
    \centering
    \includegraphics[width = \textwidth]{figures/fig1_brinkmann_experiments.pdf}
    \caption{(a) Snapshots of experiments performed by~\cite{Seemann2011JPhysCondMat} in which liquid droplets are condensed within an array of deformable microchannels. A periodic pattern of droplets emerges at relatively early times, and persists throughout as the droplets grow by further condensation. (b) Schematic diagram of the experiments shown in (a). We refer to variations in the plane parallel (perpendicular) to the channel walls as in-plane (transverse, respectively), as indicated.}
    \label{fig:Experiments}
\end{figure}


 In each of the above examples, a rigid geometry is imposed externally upon the liquid. However, new possibilities open up if walls confining the liquid are flexible; in this case, the presence of liquid can feed back on the channel geometry and thus significantly affect the behaviour of the instability. Such fluid-structure interaction can both suppress and promote instability;  the Saffman-Taylor instability, for example, may be suppressed when one of the rigid walls is replaced by a flexible one~\citep{PihlerPuzovic2012PRL, PihlerPuzovic2013JFM}, while the instability-driven clustering of submerged micropillars that occurs when liquid is evaporated relies on the deformation of the pillars~\citep{DeVolder2013Angewandte}.  Another example of deformation promoting instability can be found in the experiments of~\citet{Seemann2011JPhysCondMat}, shapshots of which are shown in figure~\ref{fig:Experiments}; it has been speculated that the weaving pattern that ultimately emerges in these experiments is the result of the growth of an interfacial instability that relies on fluid induced deformations of channel walls~\citep{BradleyPhDthesis}. In more detail: in these experiments, droplets of liquid condense at the base of a three-dimensional array of channels (shown schematically in figure~\ref{fig:Experiments}(b)); the positive Laplace pressure within the droplets (the channel walls are non-wetting) results in an outward deformation of the adjacent channel walls, which is exacerbated as the volume of liquid continues to grow via condensation. Increases in channel wall deformation promote the pattern formation, which is characterized by the presence of liquid only in isolated regions, in two ways: (1) by reducing the meniscus pressure (droplet pressure is inversely proportional to channel width) promoting flow towards the centre of the droplet, and (2) by reducing the probability of droplet nucleation away from existing droplets, where the channels are less exposed to the ambient environment. Crucially, if the walls were rigid, liquid would simply condense uniformly within the channels and there would be no instability. 

The complex contact line, as well as possible interaction between droplets in neighbouring channels, adds significant complexity to a conceptual description of this instability; in this paper, we consider a simplified setup which is shown schematically in figure~\ref{fig:mechanism_schematic}: a single channel consisting of two flexible walls and a rigid base, which is part filled with liquid. This setup retains the interaction between capillary flow and bending deformations (or `bendocapillarity'), which is a key facet of the experiments, but the removes any dependence on contact lines, as well as interaction between droplets in neighbouring channels. We also simplify the system by considering only a constant volume of liquid in the channel (this is reasonable because condensation occurs very slowly in the experiments, a statement that is justified mathematically in due course).

The mechanism for instability is in this setup is elucidated in figure~\ref{fig:mechanism_schematic}. The base state has a stationary interface that is uniform in the in-plane direction (the direction parallel to the plane of channel walls if they were undeformed, see inset in figure~\ref{fig:mechanism_schematic}). For a non-wetting liquid with a negative Laplace pressure, the channel walls are tapered outwards (figure~\ref{fig:mechanism_schematic}a). Protrusions of a perturbation to this interface will experience a relaxation in confinement, resulting from the combination of two complementary effects: (1) the outwards tapering of the channel walls prior to the perturbation, and (2) the elastic response to the perturbation. The latter is a result of the asymmetric boundary conditions (clamped at the end containing the liquid, and free at the other), which tends to amplify closer to the end containing the liquid than at the end that does not~\citep{Bradley2019PRL}. The relaxation of confinement reduces the liquid pressure at protrusions of the perturbation (the pressure is inversely proportional to the channel width); the perturbation will grow, and the instability will be amplified, if the total pressure reduction from the combination of these two effects exceeds the (stabilizing) pressure increase that results from an interface that is now locally convex in the in in-plane direction (i.e. if the wavelength of the perturbation is sufficiently long).

\begin{figure}
    \centering
    \includegraphics[width=\textwidth]{figures/fig2_mechanism_schematic.pdf}
    \caption{Schematic representation of the air-liquid interface in the bendocapillary instability for (a) non-wetting and (b) wetting liquids. In each case, the grey and black outlines indicate the configuration (channel shape and contact line) prior and post perturbation, respectively. Upon perturbing, the contact line is deformed from a straight line to a periodic curve; in both wetting and non-wetting cases, the channel experiences a deformation that enhances (reduces, respectively) the deformation of the channel walls in the base state in regions adjacent to protrusions (invaginations). The inset in (b) indicates the in-plane and transverse directions referred to throughout the main text. }
    \label{fig:mechanism_schematic}
\end{figure}

Perhaps surprisingly, a similar mechanism is also applicable to wetting configurations. In the wetting case, the bulk liquid pressure is negative, and the associated channel deformations are inwards, but the result is the same: protrusions now advance into a stronger confinement, which is further enhanced by the additional elastic response. The combination of these two effects reduces the local liquid pressure (it becomes more negative), which might cause the perturbation to grow if its wavelength is sufficiently long. Therefore, both wetting and non-wetting liquids may theoretically experience this bendocapillary instability in the same channel. This is in contrast to the closest analogy in a rigid channel, in which the walls are tapered: configurations with wetting (non-wetting, respectively) liquids are only theoretically unstable to sufficiently perturbations of sufficiently long wavelength if the channel is tapered inwards (outwards) away from the liquid.

In this paper, we investigate this instability mechanism quantitatively. We begin by presenting a simple scaling argument for the wavenumber of unstable modes and the associated growth rates, before developing a formal mathematical model of the system shown in figure~\ref{fig:mechanism_schematic}. The model equations are then non-dimensionalized; in doing so, we identify three key parameters relating to the aspect ratio of the channel, the ability of the liquid to deform it, and the amount of liquid it contains. Following this, in \S\ref{S:Equilibria}, we set out the base states (equilibria) of the system, which are parametrized by these three dimensionless numbers; these equilibria are essentially those described by~\citet{Taroni2012JFM}, but we extend this description to include non-wetting configurations. In \S\ref{S:LSA}, we consider the linear stability of these equilibria. We numerically solve the equations which must be satisfied by perturbations, and identify several important, yet generic features of these solutions: equilibria are linearly unstable to perturbations of sufficiently small wavenumber (large wavelength), with maximum growth rates that increase with both the channel bendability and amount of liquid in the channel. In \S\ref{S:Asymptotics}, we consider the limiting case of small channel deformations. Using asymptotic methods, we examine the system of equations that perturbations must satisfy in this limit, deriving a universal dispersion relation, and verifying analytically the observations of \S\ref{S:LSA}. Finally, in \S\ref{S:Conclusion}, we discuss and summarize the results our results, and provide concluding remarks.


\section{Scaling Argument}\label{S:Scaling}
%gain quantitative insight into the mechanisms introduced in the introduction
Before developing a detailed mathematical model, we seek first to gain quantitative insight into mode selection that results from the mechanism described in \S\ref{S:Introduction}. We consider the configuration shown in figure~\ref{fig:Scaling:ScalingArgument}: a section of a channel of width $\lambda$ and length $L$ contains liquid that sits at a rigid base of thickness $2H$. The walls are clamped at the rigid base, while the opposite end is open. The other two walls of the channel are narrow and flexible; they bend in response to the liquid pressure, and are characterized by a bending stiffness $B$. Here we impose the hypothetical restriction that they may only bend in the transverse direction. We also imagine a cut along the centre of each of the deformable walls (black dashed lines in figure~\ref{fig:Scaling:ScalingArgument}a), so that the two halves on either side of this cut can deform independently of one another. Since we have restricted deformations to the transverse direction, the two halves are themselves uniform in the in-plane direction, and thus any instability that appears will be a discrete square wave instability. The schematic shown in figure~\ref{fig:Scaling:ScalingArgument} corresponds to a wetting configuration, but the scaling argument set out in this section is also applicable for non-wetting configurations.

\begin{figure}
\centering
\includegraphics[width = .99\textwidth]{figures/ScalingArgument.pdf}
\caption{Schematic diagrams of a section of a flexible channel consisting of a solid base and two flexible walls, which are only permitted to bend in the transverse direction. A cut along the centre of the channel (black dashed line) allows the two halves to bend independently of one another. (a) The system is in equilibrium with the meniscus located a distance $x_m$ from the base.  (b) The equilibrium is perturbed by moving the menisci on either side of the cut a distance $\delta$; as described in the main text, if $\lambda$ is sufficiently large, this perturbation results in the flow of liquid from troughs (blue) to peaks (red) with speed $U > 0$, amplifying the perturbation.}
\label{fig:Scaling:ScalingArgument}
\end{figure}

\citet{Taroni2012JFM} considered the two-dimensional analogue of this system, in which both halves are identical, and showed that equilibrium configurations always exist when the cross-sectional volume of liquid $\Omega$ is sufficiently small. In particular, this means that equilibria always exist when $\Omega / (2HL) \ll 1$ and the cross-sectional volume of liquid is small in comparison with the cross-sectional channel volume. Here we such consider an equilibrium with a meniscus that is located a distance denoted by $x_m$ from the clamped end (figure~\ref{fig:Scaling:ScalingArgument}). The restriction to relatively small volumes means that the channel walls are not deformed significantly, and thus $h_m$ --the half-width at the meniscus -- scales with the clamped end width, i.e. $h_m \sim H$. By conservation of mass, the meniscus position is $x_m \sim \Omega / H$, and the liquid pressure is $p_m \sim -\gamma \cos \theta/H$, where $\gamma$ is the surface tension coefficient of the liquid, and $\theta$ is the constant contact angle between the liquid and channel wall (which is assumed constant).

Before we consider perturbations to this equilibrium, we need a scaling for $h_m'$, the channel slope in the transverse direction at the meniscus. By considering a cantilever beam with bending stiffness $B$, which is deformed over a length scale $x_m$, by a uniform (Laplace) pressure $-\gamma \cos \theta / H$, we find~\citep{Timoshenko1959},
\begin{equation}\label{E:Scaling:hmprimed}
h_m' \sim -\frac{\gamma \cos \theta x_m^3}{B H}.
\end{equation}

We mimic a periodic perturbation of wavenumber $k = 2\pi/\lambda$ and amplitude $\amplitude$, by considering a region of length $\lambda$ in the in-plane direction, centred around the cut (figure~\ref{fig:Scaling:ScalingArgument}). We imagine forcing one side of the cut to advance uniformly to $x_0 + \amplitude$ and the other to retreat to $x_0 - \amplitude$ (figure~\ref{fig:Scaling:ScalingArgument} b). As discussed, the transverse interfacial curvature, and thus liquid pressure, changes as a result of this perturbation: in this wetting example, the protruding half of the interface is forced into a stronger confinement by the tapering of the equilibrium configuration, and this confinement is enhanced by the elastic response of the channel to the change in meniscus position. With this `discrete' perturbation, there is no stabilizing surface tension term, which would usually come from the in-plane curvature; here, we include this contribution manually by adding a pressure penalty of $\gamma \delta k^2$ to the protruding half.

\subsection{Mode selection}
The perturbation will grow provided that the difference in liquid pressure between the two halves, $\Delta P = p_+ - p_-$, drives liquid towards the protruding half (the red half in figure~\ref{fig:Scaling:ScalingArgument}b), i.e.~when $\Delta P<0$.

To leading order in $\amplitude$, we find that
\begin{equation}\label{E:Scaling:DeltaP_preliminary}
\Delta P\sim \gamma\left(\frac{\cos \theta}{h_-} - \frac{ \cos \theta}{h_+} + \beta \amplitude k^2\right) \sim \gamma \left(\frac{\Delta h \cos \theta}{h_m^2} + \beta \amplitude k^2\right),
\end{equation}
where $\beta>0$ is an $\mathcal{O}(1)$ scaling constant and $\Delta h = h_+ - h_- $ is the difference in the channel widths at the menisci between the two halves (see figure~\ref{fig:Scaling:ScalingArgument}b). For wetting configurations, with $\cos \theta > 0$ we expect $\Delta h < 0$, while for non-wetting configurations with $\cos \theta < 0$, we expect that $\Delta h > 0$; the first term in~\eqref{E:Scaling:DeltaP_preliminary}, which represents the transverse contribution to curvature changes, is therefore always negative and thus promotes instability.

To find a scaling for $\Delta h$, we decompose it into a contribution from the meniscus advancing into a tapered channel, and a contribution from the elastic response to the perturbation:
\begin{equation}\label{E:Scaling:ChangeInH}
\Delta h = \Delta h_\text{tapering} + \Delta h_\text{elastic}.
\end{equation}

To leading order in $\amplitude$, the tapering contribution has the same scaling as a meniscus advancing into a rigid channel whose angle is set by the equilibrium configuration, i.e.
\begin{equation}\label{E:ChangeInHtapering}
\Delta h_\text{tapering} \sim \amplitude h_m' \sim -\delta \frac{\gamma  \cos \theta x_m^3}{B H},
\end{equation}
where we have used the scaling~\eqref{E:Scaling:hmprimed} for $h_m'$.

The leading order elastic contribution is found by considering a cantilever beam of length $x_m$ that is loaded with the equilibrium liquid pressure $p_m = -\gamma \cos \theta/H$ (the dry region of the beam offers no resistance to bending in this scenario~\citep{Bradley2019PRL}). If the length of this cantilever beam is then increased to $x_m + \delta$, the corresponding increase in deflection of its tip is
\begin{equation}\label{E:Scaling:ChangeInHelastic}
\Delta h_\text{elastic} \sim -\frac{p_m}{B}\left[ \left(x_m + \amplitude\right)^4 - x_m^4\right] \sim  -\amplitude \frac{\gamma \cos \theta x_m^3}{BH}.
\end{equation}

Perhaps surprisingly, both the elastic and tapering contributions to $\Delta h$ have the same scaling. Substituting these scalings into~\eqref{E:Scaling:DeltaP_preliminary} gives
\begin{equation}\label{E:Scaling:DeltaP}
\Delta P \sim -\delta \gamma \left(\frac{\gamma \cos^2 \theta}{H^2}\frac{ x_m^3}{B H} - \beta k^2\right).
\end{equation}

By balancing the terms in~\eqref{E:Scaling:DeltaP}, we obtain
a scaling for the wavenumber of unstable modes:
\begin{equation}\label{E:Scaling:CriticalWavenumber}
k_c \sim  \left(\frac{\gamma \cos^2 \theta x_m^3}{B H^3}\right)^{1/2}
\end{equation}
We expect that perturbations with wavenumber $k \lesssim k_c$ will be unstable, while those will wavenumber $k \gtrsim k_c$ will be damped. 

\subsection{Growth rates}
When $\Delta P < 0$, liquid is sucked from the invaginations into protrusions with a typical velocity $U$ (figure~\ref{fig:Scaling:ScalingArgument}b), whose scaling we now consider. To estimate this velocity scale, we note that this pressure difference acts over a length scale $\lambda = 2\pi/k$ and so lubrication theory~\citep{Leal2007} suggests that (provided $\lambda,L \gg H$)
\begin{equation}
U \sim -\frac{H^2}{\mu}\frac{\Delta P}{\lambda}\sim \frac{H^2}{\mu}  \delta \gamma k\left(\frac{\gamma \cos^2 \theta}{H^2}\frac{x_m^3}{B H} - \beta k^2\right).
\end{equation}
The corresponding flux of liquid between the two halves of the channel is
\begin{equation}\label{E:Scaling:Flux}
Q \sim \Omega U \sim\frac{H^2}{\mu}  \delta \gamma k \Omega\left(\frac{\gamma \cos^2 \theta}{H^2}\frac{ x_m^3}{B H} - \beta k^2\right),
\end{equation}
while conservation of mass for either section requires
\begin{equation}\label{E:Scaling:MassCons}
H \lambda \dd{\delta}{t} \sim Q.
\end{equation}
Combining~\eqref{E:Scaling:Flux} and~\eqref{E:Scaling:MassCons} gives a scaling for $\sigma$, the growth rate of perturbations, as
\begin{equation}\label{E:Scaling:amplitude_ode}
\sigma = \frac{1}{\delta} \dd{\delta}{t} \sim \frac{\Omega H}{\mu}\left(\frac{\gamma \cos^2 \theta}{H^2}\frac{x_m^3}{B H} - \beta k^2\right)k^2.
\end{equation}

The wavenumber of the fastest growing modes, which result in a balance of the bracketed terms in~\eqref{E:Scaling:amplitude_ode}, can be seen to scale as $k \sim k_c$, with $k_c$ as given in~\eqref{E:Scaling:CriticalWavenumber}. The corresponding growth rate of perturbations with wavenumbers of this characteristic size is
\begin{equation}\label{E:Scaling:SigmaScaling1}
\sigma_c =  \frac{1}{\delta} \dd{\delta}{t} \sim \frac{\Omega H}{\mu}\left(\frac{\gamma \cos^2 \theta}{H^2}\frac{x_m^3}{B H} - \beta k_c^2\right)k_c^2 \sim \frac{\gamma^2 \cos^2 \theta ~ x_m^7}{\mu B H^{4}},
\end{equation}
where the final scaling uses the small deformation volume scaling, $\Omega \sim x_m H$. 

While the above calculations are rough, they suggest that (1) both wetting and non-wetting configurations of sufficiently small wavenumber will be amplified, with both the fastest growing mode and corresponding growth rate symmetric under a reversal of wettability ($\cos \theta \to -\cos \theta$), and (2) that the growth rate of unstable modes has an extremely sensitive dependence on the amount of liquid in the channel (via the meniscus position $x_m$). We now turn to a more formal calculation to interrogate these observations in detail, but shall refer back to the results of this section in due course.

\section{Mathematical Model}
%describe the model
In this section, we develop a formal mathematical model of the system discussed in \S\ref{S:Scaling}. The configuration is shown in figure~\ref{fig:Modelling:Schematic}a: a narrow cell of thickness $2H$ and length $L$ extends infinitely in the $y$-direction. The channel has a rigid boundary at $x = 0$, and is free at $x = L$. The other two walls are flexible and, in the case that they were undeformed, would coincide with the planes $z = \pm H$. These channel walls are characterized by their thickness $b$, Young's modulus $E$, density $\rho_s$, and Poisson's ratio $\poisson$. We shall assume that channel walls are relatively thin ($b \ll L$), and may therefore be characterized by their bending stiffness $B = Eb^3 / [12(1-\poisson^2)]$~\citep{Timoshenko1959}, where $\poisson$ is the Poisson's ratio of the channel walls. We take $\eta = 0.5$, corresponding to incompressible walls, throughout.

%configuration description: liquid
 Liquid of viscosity $\mu$, density $\rho_l$ and surface tension $\gamma$ sits at the solid base of the channel. We assume that the liquid makes a constant contact angle $\theta$ with the channel walls (i.e. any dynamic contact angle effects are ignored). The liquid pressure induces a deformation of the channel walls; in the following sections we describe the coupled models for the flow of liquid and deformation of the channel walls, and the non-dimensionalize the resulting system of equations.

\begin{figure}
\centering
\includegraphics[width=0.99\textwidth]{figures/fig4_schematic.pdf}
\caption{(a) Schematic diagram of liquid in a narrow channel consisting of a solid, impenetrable base at $x =0$ and two flexible walls, whose mid-planes are located at $z = \pm h(x,y,t)$. The liquid makes contact with the channel walls at the contact line $x = x_m(y,t)$. The cell extends infinitely in the $y$-direction, only a section of which is shown. (b) Cross sections of the system shown in (a) in the $(x,y)$ plane (upper) and $(x,z)$ plane (lower); the latter is taken through $y = y^*$, indicated by the dashed box in (a).}
\label{fig:Modelling:Schematic}
\end{figure}

\subsection{Preliminaries and assumptions}
We assume that the configuration is symmetric about $z = 0$, and therefore only need to consider a single channel wall. Alongside our assumption that the flexible channel walls are thin in comparison with their length, this symmetry assumption means that we can characterize the channel width at time $t > 0$ entirely by the position of the mid-plane of one wall~\citep{Reddy2006}, which is denoted by $h(x,y,t)$, (we therefore use `wall' and `wall mid-plane' interchangeably).
%We also assume that the channel walls are thin in comparison with the channel half-width, $b \ll H$, so it is reasonable to consider $2h(x,y,t)$ to be the width of the cavity between the walls.

The channel is wetted over a region $0 < x < x_m(y,t)$. We assume that $x_m \gg H$ throughout, and also that variations in the flow in the $y$ direction occur on a length scale much longer than $H$, allowing us to use lubrication theory to model the liquid flow. Since the channel geometry does not provide a natural lengthscale for flow in the $y$-direction, we postpone discussion of the $y$-lengthscale until \S\ref{S:Modelling:NonDim}, and verifying this latter assumption a posteriori. With these assumptions, the contact angle between liquid and solid is approximately that measured in the $(x,z)$ plane (figure~\ref{fig:Modelling:Schematic}b, bottom panel).

Finally, we neglect the weight and inertia of both the liquid and the channel walls, as well as the line force associated with surface tension, which have been shown to be unimportant in comparison with the Laplace pressure of the liquid in similar situations~\citep{Taroni2012JFM, Bradley2019PRL, BradleyPhDthesis}.

\subsection{Liquid flow}
The fluid flow is described using lubrication theory. The evolution of the pressure field $p(x,y,t)$ and the channel width $2h(x,y,t)$ are then coupled via Reynolds' equation
\begin{equation}\label{E:Model:Liquid:Reynolds}
\ddp{h}{t} = \nabla.\left( \frac{h^3}{3\mu} \nabla p\right) \qquad \text{in}~0 < x < x_m(y,t).
\end{equation}

The free boundary of the liquid moves in response to the flux of fluid there. Since we have ignored condensation, the flux of fluid through the menisci must balance that caused by motion and thus the following kinematic condition must hold:
\begin{equation}\label{E:Model:Liquid:Kinematic}
\ddp{x_m}{t} = -\left.\frac{h^2}{3\mu}\nabla p .\frac{\mathbf{n}}{|\mathbf{n}|}\right|_{x = x_m(y,t)}
\end{equation}
Here $\mathbf{n} = \mathbf{e}_x -\partial_y x_m  \mathbf{e}_y$ is the normal to the interface in the $(x,y)$ plane (figure~\ref{fig:Modelling:Schematic}b).

According to Laplace's law, the liquid pressure adjacent to the interface is
\begin{equation}\label{E:Model:Liquid:LaplaceBC}
 p(x = x_m, y,t) = \gamma\left(C_{\perp} + C_{\parallel}\right),
\end{equation}
where $C_{\perp}$ and $C_{\parallel}$ are the transverse and in-plane interfacial curvatures, respectively. Our neglect of gravity means that the meniscus is a minimal surface: in the $(x,z)$ plane the menisci are therefore approximately arcs of circles (figure~\ref{fig:Modelling:Schematic}b) with curvatures
\begin{equation}\label{E:Model:Liquid:CPerp}
C_{\perp} = -\frac{\cos \theta}{h(x=x_m,y,t)}.
\end{equation}

Our assumption that variations in the $y$-direction occur on a length scale much larger than $H$ means that we can approximate the in-plane interfacial curvature by
\begin{equation}\label{E:Model:Liquid:CParallel}
C_{\parallel} = -\ddp{^2 x_m}{y^2}.
\end{equation}

Finally, we impose a no-flux condition at the impenetrable base:
\begin{equation}\label{E:Model:Liquid:nofluxBC}
\ddp{p}{x}=0 \qquad \text{at}~x =0.
\end{equation}

\subsection{Wall deformation}
The energy penalty associated with stretching the thin channel walls, which scales with $b/L \ll 1$, is very high in comparison with the energy penalty associated with stretching them, which scales with $(b/L)^3$~\citep{Pini2016SciRep}. It is therefore reasonable to neglect stretching of the channel walls, and model them simply as thin plates undergoing pure bending deformations under an applied load $q(x,y,t)$. The position of the channel wall mid-plane $h(x,y,t)$ therefore satisfies~\citep{Timoshenko1959}
%In the experiments shown in Figure~\ref{fig:Intro:ExptSnapshots}, the channel walls appear to undergo deformations that are large in comparison with their thickness.  We might expect, therefore, that the modelling of the channel's response to liquid pressure should include self-induced in-plane stretching~\citep{Timoshenko1959}. However, the energy penalty associated with in-plane stretching, which scales with $b/L$, is very high in comparison with the bending energy $(b/L)^3$ for thin flexible objects~\citep{Pini2016SciRep}. We therefore ignore stretching, and the channel walls can therefore be modelled as thin plates undergoing pure bending deformations under an applied load $q(x,y,t)$. The position of the channel wall mid-plane $h(x,y,t)$ therefore satisfies~\citep{Timoshenko1959}
\begin{equation}\label{E:Model:Wall:Bilaplacian}
B\nabla^4 h = q.
\end{equation}

The channel deformation is coupled to the liquid pressure via the applied load:
\begin{equation}\label{E:Model:Wall:BilaplacianPressure}
q(x,y,t) = \left\{\begin{array}{ll}  p(x,y,t) & 0 < x< x_m(y,t),\\ 0 & x_m(y,t) < x < L.
\end{array}\right.
\end{equation}

To close the problem, we require boundary conditions at the channel ends, $x= 0$ and $x = L$, as well as at the interface, $x = x_m$. We apply a straightforward clamped condition at $x = 0$,
\begin{equation}\label{E:Model:Wall:ClampedBC}
h= H, \quad \ddp{h}{x} = 0, \qquad \text{at}~x = 0.
\end{equation}

The channel walls are free at $x = L$; for a thin plate undergoing pure bending deformations, free boundary conditions are imposed by requiring~\citep{Timoshenko1959}
\begin{equation}\label{E:Model:Wall:FreeEndBC}
\ddp{^2 h}{x^2} + \poisson \ddp{^2 h}{y^2} =0, \quad  \ddp{^3 h}{x^3} + (2-\poisson) \ddp{^3 h}{x \partial y^2} = 0\quad \text{at}~x = L.
\end{equation}

%Comment on the presence of y derivatives meaning we can't integrate out the dry region, in general?
The boundary condition~\eqref{E:Model:Wall:FreeEndBC} applies only when the channel walls do not touch. If the channel walls touch -- as might be envisaged in the wetting case, where the walls are drawn towards one another -- the boundary condition~\eqref{E:Model:Wall:FreeEndBC} must be modified to include a repulsive shear force. This touching ends scenario requires significant deformations, which are not commensurate with the relatively small volumes that are of primary interest here; we therefore assume that~\eqref{E:Model:Wall:FreeEndBC} holds throughout. 

At the meniscus, we assume that the channel and its slope, as well as the moments and shear forces it supports are continuous:
\begin{align}
\left[ h \right]_{-}^{+}  &= 0 & &\text{[continuous channel shape]},\label{E:Model:Wall:ContinuityBC1}\\
\left[\ddp{h}{x} \right]_{-}^{+} = \left[\ddp{h}{y} \right]_{-}^{+} &=0 & &\text{[continuous channel slope],}  \\ 
\left[\ddp{^2 h}{x^2} + \poisson \ddp{^2 h}{y^2} \right]_{-}^{+} =   \left[\ddp{^2 h}{y^2} + \poisson \ddp{^2 h}{x^2}  \right]_{-}^{+} =  \left[\ddp{^2 h}{x \partial y} \right]_{-}^{+} &= 0 & &\text{[continuous bending moments],} \\
\left[\ddp{^3 h}{x^3} + \ddp{^3 h}{ x\partial y^2}\right]_{-}^{+} =  \left[\ddp{^3 h}{ x^2 \partial y} + \ddp{^3 h}{y^3} \right]_{-}^{+} &=0  & &\text{[continuous shear forces].}\label{E:Model:Wall:ContinuityBC4} 
\end{align}
Here $\left[ f \right]_-^+ = f(\x_m^+,y,t) - f(\x_m^-,y,t)$ denotes the jump in the quantity $f$ across $x = x_m$, and has both $y$- and $t$-dependence in general.

In summary, the system is described by the liquid pressure $p$ and wall deflection $h$, which satisfy the coupled PDEs~\eqref{E:Model:Liquid:Reynolds} and~\eqref{E:Model:Wall:Bilaplacian}--\eqref{E:Model:Wall:BilaplacianPressure} for the fluid flow and wall deformation, respectively. This system of PDEs is to be solved alongside the kinematic condition conditions~\eqref{E:Model:Liquid:Kinematic}, the no-flux condition~\eqref{E:Model:Liquid:nofluxBC}, the clamped end condition~\eqref{E:Model:Wall:ClampedBC}, the free end condition~\eqref{E:Model:Wall:FreeEndBC}, and the continuity conditions~\eqref{E:Model:Wall:ContinuityBC1}--\eqref{E:Model:Wall:ContinuityBC4}. We now turn to a non-dimensionalization of this system.


\subsection{Non-dimensionalization}\label{S:Modelling:NonDim}
The system of model equations is non-dimensionalized by scaling variables appropriately. Variations in the $x$- and $z$-directions have natural length scales set by the channel geometry, and corresponding dimensionless variables (denoted by hats)
\begin{equation}\label{E:Modelling:NonDim:SpatialScaling}
\hat{h} = \frac{h}{H}, \qquad \hat{x} = \frac{x}{L}, \qquad \hat{x}_m = \frac{x_m}{L}.
\end{equation}

The channel geometry does not provide a length scale for variations in the $y$-direction, and so, for now, we choose the scale $L$ used in the $x$-direction, introducing
\begin{equation}\label{E:Modelling:NonDim:yscaling}
\hat{y} = \frac{y}{L}.
\end{equation}
Note that in the scaling argument of \S\ref{S:Scaling}, we identified a critical wavelength for instability in the $y$-direction,
\begin{equation}\label{E:Modelling:NonDim:LengthscaleFromScaling}
L_y = \left(\frac{B H^3}{\gamma  \cos^2 \theta L^3}\right)^{1/2},
\end{equation}
and so we anticipate the appearance of the dimensionless wavenumber
\begin{equation}\label{E:Modelling:NonDim:ExpectedWavenumber}
L/L_y =  \left(\frac{\gamma \cos^2 \theta L^5}{B H^3}\right)^{1/2}
\end{equation}
in our stability analysis.

Time and pressure variables are non-dimensionalized by scaling with a capillary time scale and a bending pressure scale, respectively,
\begin{equation}\label{E:Modelling:NonDim:TimeAndPressureScaling}
\hat{t} =\frac{t}{ \tau_c} =  \frac{|\gamma \cos \theta| H}{\mu L^2}t, \qquad \hat{p} = \frac{L^4}{B}p.
\end{equation}
Using values from~\citet{Bradley2019PRL}, who studied a similar bendocapillary system, the typical capillary timescale $\tau_c$ is on the order of 10~s; this is significantly longer than the timescale on which the channels in the motivating experiments (figure~\ref{fig:Experiments}a) fill owing to condensation, which is on the order of minutes. The quasistatic picture considered here is therefore appropriate for these experiments.

After scaling variables in this way, the PDE system~\eqref{E:Model:Liquid:Reynolds} and~\eqref{E:Model:Wall:Bilaplacian}--\eqref{E:Model:Wall:BilaplacianPressure} becomes
\begin{align}
\ddp{\hat{h}}{\hat{t}} &=\frac{1}{3|\nu|}\hat{\nabla}. \left[\hat{h}^3\hat{\nabla}\hat{p}  \right] & & 0 <  \hat{x} < \hat{x}_m(\hat{y},\hat{t}),\label{E:Modelling:NonDim:PDE1}\\
  \hat{p}&=0 & &
\hat{x}_m(\hat{y},\hat{t}) < \hat{x} < 1,\label{E:Modelling:NonDim:PDE2}\\
\hat{p} &=\hat{\nabla}^4 \hat{h}& &0 < \hat{x} < 1.\label{E:Modelling:NonDim:PDE3}
\end{align}
Here
\begin{equation}\label{E:Modelling:NonDim:Bendability}
\bendability = \frac{\gamma \cos \theta L^4}{B H^2}
\end{equation}
is the channel `bendability', which characterizes the ability of the typical capillary pressure within the liquid to bend the channel walls: large $|\nu|$ indicates that the channel walls are easily deformed (achieved by having low bending stiffness or high surface tension, for example), and vice versa for small $|\nu|$. The sign of $\nu$ captures the wetting conditions: $\nu > 0$ corresponds to wetting conditions ($\cos \theta >0$), while $\nu < 0$ corresponds to non-wetting conditions ($\cos \theta >0$).

Note that the sixth order PDE that results from combining~\eqref{E:Modelling:NonDim:PDE1}--\eqref{E:Modelling:NonDim:PDE2} is a two-spatial-dimension analogue to PDEs that often appear in studies of elastocapillary dynamics~\citep[see][for example]{Flitton2004EJApplMech, Duprat2011JFM, Aristoff2011IntJNonlinMech}.

After non-dimensionalizing, the boundary conditions~\eqref{E:Model:Wall:ClampedBC}--\eqref{E:Model:Wall:FreeEndBC} and~\eqref{E:Model:Wall:ContinuityBC1}--\eqref{E:Model:Wall:ContinuityBC4} on the channel wall position become
\begin{equation}\label{E:Modelling:NonDim:ClampedBC}
\hat{h} = 1, \quad \ddp{\hat{h}}{\hat{x}} = 0 \quad \text{at}~ \hat{x} = 0,
\end{equation}
\begin{equation}\label{E:Modelling:NonDim:FreeBC}
\ddp{^2 \hat{h}}{ \hat{x}^2}  + \poisson \ddp{^2 \hat{h}}{\hat{x}^2} = 0 \quad  \ddp{^3 \hat{h}}{ \hat{x} ^3}  + (2-\poisson) \ddp{^3 \hat{h}}{\hat{x}\partial \y^2}=0 \quad \text{at}~ \hat{x} = 1,
\end{equation}
\begin{align}\label{E:Modelling:NonDim:ContinuityBC1}
\left[ \hat{h} \right]_{-}^{+} = \left[\ddp{\hat{h}}{\hat{x}}\right] _{-}^{+}  = \left[\ddp{\hat{h}}{\hat{y}} \right]_{-}^{+} &= 0 ,\\
 \left[\ddp{^2 \hat{h}}{\hat{x}^2} +\poisson \ddp{^2 \hat{h}}{\hat{y}^2}  \right]_{-}^{+} = \left[\ddp{^2 \hat{h}}{\hat{y}^2} +\poisson \ddp{^2 \hat{h}}{\hat{x}^2}  \right]_{-}^{+}  = \left[\ddp{^2 \hat{h}}{\hat{x} \partial \hat{y}}  \right]_{-}^{+}  &= 0, \label{E:Modelling:NonDim:ContinuityBC2}
\\
\left[\left(\ddp{^3 \hat{h}}{\hat{x}^3} + \ddp{^2 \hat{h}}{\hat{x} \partial \hat{y} ^2}\right) \right]_{-}^{+}  = \left[ \left(\ddp{^2 \hat{h}}{\hat{x}^2 \partial y} + \ddp{^2 \hat{h}}{\hat{y} ^3}\right) \right]_{-}^{+} &=0 \label{E:Modelling:NonDim:ContinuityBC3},
\end{align}
where the jump conditions in~\eqref{E:Modelling:NonDim:ContinuityBC1}--\eqref{E:Modelling:NonDim:ContinuityBC3} are now (and henceforth) evaluated across the dimensionless meniscus position $\hat{x}_m$.

The boundary conditions~\eqref{E:Model:Liquid:LaplaceBC} and~\eqref{E:Model:Liquid:nofluxBC} on the pressure become
\abeqn{E:Modelling:NonDim:PressureJumpBC}{
\ddp{\hat{p}}{\hat{x}} = 0 \quad \text{at}~\hat{x} = 0, \qquad \quad \hat{p} = -\nu\left(\frac{1}{\hat{h}} + \aspect \ddp{^2 \hat{x}_m}{\hat{y}^2}\right)\quad \text{at}~\hat{x} = \hat{x}_m.}
Here, $\aspect = H/(L \cos \theta)$ is a reduced channel aspect ratio, which arises as the ratio between the typical radii of curvature in the transverse and in-plane directions. Note that $\aspect$ always has the same sign as $\nu$. In particular, this means that their quotient $\nu / \aspect$, which shall appear frequently, is always positive.

We retain the final term in~\eqref{E:Modelling:NonDim:PressureJumpBC} despite it being higher order in $\aspect \ll 1$ (we assume $\cos \theta \sim \mathcal{O}(1)$); for periodic perturbations with wavenumber $k$, this term is $\mathcal{O}(\aspect k^2)$ and will become important for large wavenumber (short wavelength) perturbations. (Note also that the final term in~\eqref{E:Modelling:NonDim:PressureJumpBC} is larger than the errors introduced by using lubrication theory, which are $\mathcal{O}(\aspect^2)$.)

Finally, the dimensionless meniscus position $\hat{x}_m = \hat{x}_m(\hat{y}, \hat{t})$ evolves according to
\begin{equation}\label{E:Modelling:NonDim:Kinematic}
\ddp{ \hat{x} _m}{\hat{t}} = -\left.\frac{\hat{h}^2}{3|\nu|}\left(\ddp{\hat{p}}{\hat{x}} - \ddp{\hat{x}_m}{\hat{y}}\ddp{\hat{p}}{\hat{y}} \right)  \right|_{\hat{x} = \hat{x}_m},
\end{equation}
correct to $\mathcal{O}(\aspect^2)$. Henceforth, we drop hats and all variables are assumed dimensionless. 


\section{Equilibria}\label{S:Equilibria}
%set out equilibria for both wetting and non-wetting cases
As mentioned, \citet{Taroni2012JFM} considered the $y$-invariant analogue of the system~\eqref{E:Modelling:NonDim:PDE1}--\eqref{E:Modelling:NonDim:Kinematic}, and set out the conditions under which equilibria may exist for wetting conditions ($\nu > 0$). These equilibria, extended infinitely in the $y$-direction, are also equilibrium configurations in our system. In this section we briefly describe these equilibria. This description is appropriate also for non-wetting configurations. We also discuss the stability of equilibria to perturbations that are uniform in  the $y$-direction, which facilitates our understanding of the behaviour for perturbations of arbitrary wavelength that follows in \S\ref{S:LSA}. 

We denote the equilibrium meniscus position and channel shape by $x_m = x_0$ and $h = h_e(x)$, respectively, suppressing the $y$-dependence to reflect uniformity in this direction. Following~\citet{Taroni2012JFM}, we use the meniscus position, $x_0$, to parametrize the wall shapes and the cross sectional volume,
\begin{equation}\label{E:Equilibria:Volume}
V  = V(x_0) = \int_{0}^{x_0} h_e(x)~\mathrm{d}x.
\end{equation}
Taroni and Vella showed that the equilibria have wall shapes
\begin{equation}\label{E:Equilibria:EqShape}
\h_e(\x) = \left\{\begin{array}{l l}
 \h_m+ \frac{\nu}{24 \h_m}\left[4 x_0^3( \x_0 - \x) - ( \x_0 - \x)^4\right] &  0 < \x <  \x_0, \\
 \h_m -\frac{\nu }{6 h_m}(\x - \x_0) \x_0^3 & \x_0 < \x < 1.
\end{array}\right.
\end{equation}

The uniform pressure associated with~\eqref{E:Equilibria:EqShape} is $p = p_0 = -\nu/h_0$, where $h_0 \coloneqq  \h_e(\x_0)$. To satisfy the pressure boundary condition~\eqref{E:Modelling:NonDim:PressureJumpBC}b and the volume constraint~\eqref{E:Equilibria:Volume}, $x_0$ and $h_0$ must satisfy
\abeqn{E:Equilibria:MeniscusDispQuadratic}{
\h_0^2 - \h_0 + \frac{\nu \x_0^4 }{8} = 0, \quad \text{and}\quad V= \x_0 \h_0 + \frac{3\nu }{40 \h_0 x_0^5}, 
}
respectively. In addition, to avoid situations in which the walls touch, we require
\begin{equation}\label{E:Equilibria:NoTouchCond}
\h_e(\x = 1) = \h_0 - \frac{\nu  (1- \x_0 )\x_0^3}{6 \h_0} > 0.
\end{equation}

Note that equation~\eqref{E:Equilibria:MeniscusDispQuadratic}a has roots
\begin{equation}\label{E:Equilibria:MeniscusWidth}
h_0 = h_0^{\pm} = \frac{1}{2}\left[ 1 \pm \left(1 - \frac{\nu x_0^4}{2}\right)^{1/2}\right].
\end{equation}
so that a maximum of two equilibria with the same parameter values may exist. For now we distinguish between the these two possibilities by referring to them as `$+$' and `$-$' roots according to the sign taken in~\eqref{E:Equilibria:MeniscusWidth}, but we shall ultimately only be interested in the `$-$' root.

Non-wetting configurations ($\nu < 0$) always have only a single equilibrium: the `$-$' root of~\eqref{E:Equilibria:MeniscusWidth} exists for all $V$, while the `$+$' root has $h_0 < 0$ for all $V$, which is nonphysical. The channel width at the free end, $h_e(x = 1)$, corresponding to this root, increases monotonically with volume $V$ (figure~\ref{fig:Equilibria}b). %It is not immediately obvious that this should be the case, since increases in volume result in liquid pressure applied over a larger area, which acts to increase the deflection, while the associated reduction in liquid pressure acts in the opposite direction.

In contrast, for wetting configurations ($\nu >0$), there are regions in which each of none, one, or both of the roots of~\eqref{E:Equilibria:MeniscusWidth} correspond to physically realistic equilibria, depending on the volume (figure~\ref{fig:Equilibria}a). \citet{Taroni2012JFM} described in detail the location of these equilibria; here we simply note that (i) the `$-$' root only exists for a narrow band of sufficiently large $V$, with the no-touching~\eqref{E:Equilibria:NoTouchCond} condition violated at the lower $V$ end of this band (figure~\ref{fig:Equilibria}b), and (ii) the `$+$' root exists for all $V$ up to some finite value (for $\nu = 5$, this value is approximately 0.68, see figure~\ref{fig:Equilibria}b).

\begin{figure}
\centering
\includegraphics[width =0.98\textwidth]{figures/fig5_equilibria.pdf}
\caption{(a)--(b) Free end channel width $h_e(x = 1)$ in steady solutions of the model equations~\eqref{E:Modelling:NonDim:PDE1}--\eqref{E:Modelling:NonDim:Kinematic} with (a) $\nu = 5$ and (b) $\nu = -5$. Where appropriate, the equilibria corresponding to the `$+$' and `$-$' roots in~\eqref{E:Equilibria:MeniscusWidth} are indicated by red and blue curves, respectively (the latter do not exist in the non-wetting case). (c)--(d) Growth rate $\sigma_u$ of uniform perturbations (i.e.~of the form~\eqref{E:Equilibria:uniform_perturbation}) to the equilibria corresponding to those shown in (a) and (b), respectively. Each equilibria is associated with two values of $\sigma_u$, one of which is always zero. The inset in (c) is a close up of the section of the main figure indicated by the black dashed box.}\label{fig:Equilibria}
\end{figure}

Before moving on to consider the stability of these equilibria to in-plane perturbations, we briefly consider their stability to \textit{uniform} perturbations (or, equivalently to in-plane perturbations with zero wavenumber). In this scenario, the instability mechanism is somewhat simpler than that described in \S\ref{S:Introduction}: the meniscus would like to advance to wet the beams over a longer length, but the additional deformation that results will incur a bending energy penalty.

Following~\citet{Taroni2012JFM}, we probe the stability of the equilibria by substituting
\begin{equation}\label{E:Equilibria:uniform_perturbation}
h = h_e(x) + \varepsilon \Lambda_u(x)\exp(\sigma_u t),\quad p = p_0 + \varepsilon \Pi_u(x)\exp(\sigma_u t),\quad x_0 = x_0 +  \varepsilon \exp(\sigma_u t),
\end{equation}
where $\varepsilon \ll 1$ is arbitrary, into the model equations~\eqref{E:Modelling:NonDim:ClampedBC}--\eqref{E:Modelling:NonDim:Kinematic}; linearizing in $\varepsilon$ yields a boundary value problem (BVP) which can be solved numerically to obtain the growth rate $\sigma_u$ for a given $\nu$ and $V$; here, and in the following, numerical solutions are obtained using the BVP4c routine implemented in \textsc{matlab}~\citep{Kierzenka2001BVP}. The code used to solve this system of equations can be found at ref.~\citet{BendocapillaryRepo}.

For each of the roots of~\eqref{E:Equilibria:MeniscusWidth}, this boundary value problem has two solutions, resulting in two unique values of $\sigma_u$ for each equilibrium (figure~\ref{fig:Equilibria}c--d). We find that one of these growth rates is always zero, and refers to a uniform advance of the meniscus with no corresponding change in channel shape. This solution corresponds to a situation in which mass is not conserved in sections through the $(x,z)$ plane, but we do not rule out this ostensibly non-physical situation in the following because it is reasonable that liquid can be recruited from the $y$-direction, when variations in this direction are considered.

The `$-$' root, which only exists in the wetting case, has a non-zero growth rate that is always positive: these equilibria are unstable to uniform perturbations. In contrast, the `$+$' root, which exists for both wetting and non-wetting configurations, has a non-zero growth rate this negative: these equilibria are always stable. In addition, for small volumes, these growth rates are very negative. In this case, as $V$ increases, so too does the growth rate: the bending penalty incurred by the perturbation is reduced when the meniscus is closer to the free end, where the channel is effectively softer.

Since we are primarily interested in small volume configurations, we shall henceforth ignore the equilibria corresponding to the `$-$' root in~\eqref{E:Equilibria:MeniscusWidth}.

\section{Linear Stability Analysis}\label{S:LSA}
%set out linearized equations and present numerical solutions of them 
In this section, we consider the linear stability of the equilibria described in the previous section to in-plane perturbations with an \textit{arbitrary} wavenumber $k$. We do so by setting
\begin{align}
h &= h_e(x) + \varepsilon \Lambda(x)\exp(\sigma t + i k y),\label{E:LSA:Perturbation1} \\
 p &= p_0 + \varepsilon \Pi(x)\exp(\sigma t + i k y),\\ x_m  &= x_0 +  \varepsilon \exp(\sigma t + i k y),\label{E:LSA:Perturbation3}
\end{align}
with $\varepsilon$ again arbitrary, in the model equations~\eqref{E:Modelling:NonDim:ClampedBC}--\eqref{E:Modelling:NonDim:Kinematic}. Linearizing the resulting problem yields the following BVP for $\Lambda$ and $\Pi$:
\begin{align}
\sigma \Lambda &=  \frac{h_e^2}{3|\nu|}\left[3\dd{h_e}{x} \dd{\Pi}{x} + h_e\left(\dd{^2 \Pi}{x^2} - k^2 \Pi\right)\right] & &0 < x < x_0,\label{E:LSA:ODEwet}\\
0 &= \Pi & &x_0 < x < 1,\label{E:LSA:ODEdry}\\
\Pi &= \dd{^4 \Lambda}{x^4} - 2k^2 \dd{^2 \Lambda}{x^2} + k^4 \Lambda & & 0 < x <1.\label{E:LSA:pressure2shape}
\end{align}
The boundary conditions on~\eqref{E:LSA:ODEwet}--\eqref{E:LSA:pressure2shape} are
\begin{align}
\Lambda &= 0 = \dd{\Lambda}{x} = \dd{\Pi}{x} & &\text{at}~x = 0,\label{E:LSA:BC_at_0}\\
\Pi &= \frac{\nu}{h_m^2}\left(\dd{h_e}{x} + \Lambda\right) + \nu \aspect k^2 & &\text{at}~x = x_0,\label{E:LSA:pressure_bc}\\
\dd{^2 \Lambda}{x^2} - \poisson k^2
\Lambda &= 0 = \dd{^3 \Lambda}{x^3} - (2-\poisson)k^2 \dd{\Lambda}{x} & &\text{at}~x = 1,\label{E:LSA:BC_at_1}
\end{align}
\begin{equation}\label{E:LSA:jump_conds}
\left[\Lambda\right]_-^+= \left[\dd{\Lambda}{x}\right]_-^+ = \left[\dd{^2\Lambda}{x^2}\right]_-^+= 0, \quad\left[\dd{^3 \Lambda}{x^3}\right]_-^+ = \frac{\nu}{h_m }.
\end{equation}
The growth rate $\sigma$ satisfies
\begin{equation}\label{E:LSA:kinematic}
\sigma = -\frac{h_m^2}{3|\nu|}\left.\ddp{\Pi}{x}\right|_{x = x_0}.
\end{equation}

The terms on the right hand side of~\eqref{E:LSA:pressure_bc} elucidate the mechanism described in \S\ref{S:Introduction}: from left to right, they correspond to the transverse curvature changes arising from channel tapering set by the base state, transverse curvature changes arising from the elastic response to the perturbation, and the stabilizing term that arises from the locally convex in-plane curvature of the interface.


\subsection{Numerical Results}
As for the uniform perturbation discussed in the previous section, we find two unique numerical solutions of~\eqref{E:LSA:ODEwet}--\eqref{E:LSA:kinematic}, and which is of these is returned depends on the proximity to the initial guess for $\sigma$. The two branches originate at $k =0$ from the $\sigma_u$ discussed in the previous section. Here we are interested only in the root that originates from the zero solution for $\sigma_u$: this branch shows positive growth rates (figure~\ref{fig:LinearStability:GrowthRates}a), while the branch that originates from the non-zero solution for $\sigma_u$ has very high decay rates even for small $k$~\citep[not shown, see][]{BradleyPhDthesis}. Henceforth, we use $\sigma(k)$ for the branch that originates from the zero solution for $\sigma_u$ (i.e. $\sigma(k=0) = 0$) that is of interest.



 %In particular, those solutions on the $\sigma_1$ branch originate from the $k = 0$ solution that does not conserve mass in the $x$-direction; while this is nonphysical for two-dimensional equilibria, in three-dimensions for $k \neq 0$, however, it is possible to move mass in the $y$-direction and there is no contradiction.

\begin{figure}
\centering
\includegraphics[width =\textwidth]{figures/fig6_growth_rates.pdf}
\caption{(a) Growth rates $\sigma$, and (c) channel shape perturbations, $H(x_0)$, obtained by numerically solving the boundary value problem~\eqref{E:LSA:ODEwet}--\eqref{E:LSA:kinematic} that arises from periodic perturbations with wavenumber $k$ to an equilibrium with cross-sectional volume $V$ (values indicated by the legend). All data correspond to solutions with either $\nu = 5$, $a = 0.01$ (solid lines) or $\nu = -5$, $a = -0.01$ (dashed lines). (b) Zoom in on the dashed box in (a), plotted on logarithmic axes. Solutions for $\nu = 5$ and $\nu = -5$ are almost indistinguishable in (b).}
\label{fig:LinearStability:GrowthRates}
\end{figure}


The salient observations from the dispersion relations shown in figure~\ref{fig:LinearStability:GrowthRates} are, firstly, that $\sigma > 0$ for sufficiently small wavenumbers, an observation appears to be generic in numerical solutions of the BVP~\eqref{E:LSA:ODEwet}--\eqref{E:LSA:kinematic} for both wetting and non-wetting configurations. Secondly, $\sigma\sim k^2$ as $k \to 0$ (figure~\ref{fig:LinearStability:GrowthRates}b) as suggested by the scaling argument~\eqref{E:Scaling:amplitude_ode}. (We shall derive this formally in the case of small deformations shortly.)

Going beyond these observations, we see that numerical solutions also indicate that growth rates are not symmetric in $\nu \to -\nu$, with growth rates marginally larger than wetting configurations than non-wetting configurations with the same $|\nu|$ and $|\aspect|$ (figure~\ref{fig:LinearStability:GrowthRates}a). It is not straightforward to provide a causal link between a reversal of wetting conditions ($\nu \to -\nu$, $a \to -a$) and changes in the growth rate, since the pressure gradient and equilibrium meniscus displacement (which set the growth rate via~\eqref{E:LSA:kinematic}) are intimately coupled in the BVP~\eqref{E:LSA:ODEwet}--\eqref{E:LSA:kinematic}. Note, however, that this asymmetry disappears in the limit $V \to 0$; in the following section, we show this explicitly in the case of small deformations (a case that covers the limit $V \to 0$) but this result can be seen by noting that for $V$, $h_e(x)\approx 1$ and $h_e'(x) \approx 0$, in which case the BVP is anti-symmetric in $\nu \to -\nu$, $a \to -a$; the growth rate, which depends on the product of $P$ and $H$ (and their derivatives), will therefore be invariant under this transformation. %\blue{I think we should also be able to say qualitatively why the difference disappears as $k \to 0, \infty$, but it doesn't seem immediate to me from~\eqref{E:LSA:ODEwet}--\eqref{E:LSA:kinematic}.}

%discussion of stability and comparison between wetting and non-wetting cases
Another generic feature of numerical solutions of the BVP~\eqref{E:LSA:ODEwet}--\eqref{E:LSA:kinematic} is that $H(x_0)$ --the perturbation to the shape at the meniscus -- is negative for wetting configurations and positive for non-wetting configurations (figure~\ref{fig:LinearStability:GrowthRates}c). In other words, the channel deformation of the base state is enhanced at protrusions and reduced at invaginations for both wetting and non-wetting conditions, confirming the suggestion made in the scaling argument in \S\ref{S:Scaling}. In the results shown in figure~\ref{fig:LinearStability:GrowthRates}c, the channel shape perturbation tends to increase in magnitude with the volume $V$ and decrease with the wavenumber $k$; in the case of small deflections, however, the channel shape perturbation at the meniscus is, perhaps surprisingly, independent of the wavenumber, as we shall see. 

For the parameter values used in figure~\ref{fig:LinearStability:GrowthRates}, the fastest growing mode, denoted $k^*$, and the corresponding growth rate $\sigma^* = \sigma(k^*)$, both increase with cross-sectional volume $V$. This is in qualitative agreement with the scaling argument~\eqref{E:Scaling:SigmaScaling1}, which, after non-dimensionalizing with the length scale $L$ and time scale $\tau_c$, suggests the typical scales
\begin{equation}\label{E:LSA:DimensionlessScalings}
k_c  = \left(\frac{\nu V^3}{a}\right)^{1/2}, \qquad \sigma_c = \frac{\nu^2 V^7}{|a|}.
\end{equation}

For a quantitative assessment of the scaling argument, we show numerically obtained dispersion relations $\sigma(k)$, rescaled according to~\eqref{E:LSA:DimensionlessScalings} in figure~\ref{fig:CollapsedGrowthRates}a. Here we see that the data collapse onto a universal curve for $\nu \ll 1$ (indicated by dark shades in figure~\ref{fig:RescaledGrowthRates}), but deviate for $\nu \gg 1$ (light shades). In addition, the deviation occurs sooner (i.e. at lower $\nu$ values) for larger volume $V$ where the base state deformation is expected to be larger. To predict the envelope enclosing the curves $\sigma/\sigma_c$, and elucidate the small deformation results alluded to above, we now present an asymptotic expansion of the BVP~\eqref{E:LSA:ODEwet}--\eqref{E:LSA:kinematic} in the limit of small deformations.

\begin{figure}
\centering
\includegraphics[width = \textwidth]{figures/fig7_rescaled_growth_rates.pdf}
\caption{Numerically obtained dispersion relations $\sigma(k)$ rescaled according to~\eqref{E:LSA:DimensionlessScalings} for cross sectional volumes (a) $V = 0.1$, (b) $V = 0.2$, and (c) $V = 0.3$ with $\aspect = 0.01$ in each case. Each curve corresponds to a different value of $\nu$, taking logarithmically spaced values between $10^{-2}$ and $10^{2}$ as indicated by the colorbar. Here we show only wetting configurations ($\nu >0$, $a >0$), but non-wetting configurations behave qualitatively the same (see figure~\ref{fig:LinearStability:GrowthRates}). }
\label{fig:RescaledGrowthRates}
\end{figure}

\section{Asymptotic Analysis}\label{S:Asymptotics}
\newcommand{\param}{\delta} %small param used in this section
%use asymptotics to describe the envelope of solutions
In this section, we present an asymptotic analysis of the boundary value problem~\eqref{E:LSA:ODEwet}--\eqref{E:LSA:kinematic}  in the case of small deformations. The main result of this section is that we are able to express the envelope of growth rate curves in figure~\ref{fig:RescaledGrowthRates} analytically, and thus predict the fastest growing mode and corresponding growth rate in the limit of small deflection. Along the way, we elucidate the small volume observations mentioned in the previous section.

Small deformations are encoded by restricting ourselves to those equilibria with $\epsilon  = |\nu| V^4 \ll 1$. To begin with we note that using the volume constraint on equilibria~\eqref{E:Equilibria:MeniscusDispQuadratic}b, the volume and meniscus position of the equilibrium are related by
\begin{equation}\label{E:Asymptotics:EqMeniscusPositionExpansion}
    x_0 = V\left[1 + \mathrm{sign}(\nu) \frac{\epsilon}{20} + \order{\epsilon^2}\right],
\end{equation}
in the limit $\epsilon \to 0$. In addition, for $x < x_0$, equilibrium channel shape is given by
\begin{equation}\label{E:Asymptotics:EqChannelShapeExpansion}
    h_e(x) = 1 + \epsilon \mathrm{sign}(\nu)\psi\left(\frac{x}{V}\right) + \order{\epsilon^2}, \qquad \psi(s) = \frac{1}{24}\left\{\left[4(1-s) - (1-s)^4\right]-3\right\}.
\end{equation}

Before we can proceed with an asymptotic expansion, we must determine the size of the terms in which the wavenumber $k$ appears. Motivated by the scaling of \S\ref{S:Scaling}, in which the fastest growing modes were those with wavenumbers such that the destabilizing transverse and stabilizing in-plane curvature contributions are comparable (i.e. when the terms on the right hand side of~\eqref{E:LSA:pressure_bc} are comparable), we introduce a scaled wavenumber
\begin{equation}\label{E:Asymptotics:RescaledWavenumber}
k = \left(\frac{\nu V^3}{a}\right)^{1/2} K.
\end{equation}
In addition, in anticipation of bending deformation being primarily confined to the wet region $0 < x < x_0 = V + \mathcal{O}(\epsilon)$, we introduce the rescaled spatial variable $X = x/V$. 

After inserting~\eqref{E:Asymptotics:RescaledWavenumber} and the scaled variable $X$ into the BVP~\eqref{E:LSA:ODEwet}--\eqref{E:LSA:kinematic}, the parameter
\begin{equation}\label{E:Asymptotics:}
\delta= \frac{\nu V^5}{|a|} = \frac{\epsilon V }{|a|}
\end{equation}
naturally emerges. The parameter $\delta$ describes the relative sizes of increases in-plane and transverse bending energies of an equilibrium configuration that is subject to a perturbation with wavenumber $k \sim k_c = \sqrt{\nu V^3 /a}$. To see this we note that the typical (dimensional) in-plane and transverse wall curvatures induced by this perturbation are
\begin{equation}
\kappa_{\text{in-plane}} \sim k_c^2 \Delta h \sim \frac{\nu V^3}{a} \Delta h \quad \text{and}\quad \kappa_{\text{transverse}} \sim \frac{\Delta h}{x_0^2} \sim  \frac{\Delta h}{V^2},
\end{equation}
where $\Delta h$ is the typical change in channel thickness that results from the perturbation. The corresponding dimensionless bending energies are
\begin{align}
E_{\text{in-plane}} &\sim \frac{\left(\kappa_{\text{in-plane}}\right)^2}{k_c} \sim \left( \frac{\nu V^3}{a}\right)^{3/2}(\Delta h)^2,\\ 
E_{\text{transverse}} &\sim x_0 \left(\kappa_{\text{transverse}}\right)^2 \sim \frac{(\Delta h)^2}{V^3},
\end{align}
whose ratio is
\begin{equation}
\frac{E_{\text{in-plane}}}{ E_{\text{transverse}}}
\sim \delta^{3/2}.
\end{equation}

To make progress, we consider the limit $\delta \to 0$, which corresponds to relatively small in-plane bending deformations, compared to transverse bending deformations.  We  also  rescale  the  perturbed  channel  shape $H$ and  pressure $P$ with $\nu V^3 $to  reflect the leading order behaviour: a shearing force of magnitude $\nu$ applied over a length equal to  the  magnitude  of  the  perturbation  (the  shear  is  the  third  derivative  of  the  channel deformation, which combined with the length rescaling is responsible for the $V^3$).

After inserting~\eqref{E:Asymptotics:EqMeniscusPositionExpansion}--\eqref{E:Asymptotics:EqChannelShapeExpansion} into the BVP~\eqref{E:LSA:ODEwet}--\eqref{E:LSA:kinematic}, the problem for the rescaled channel shape and pressure,
\begin{equation}
G(X) = \frac{H(X)}{\nu V^3}, \qquad Q(X) = \frac{P(X)}{\nu V^3}
\end{equation}
respectively, as well as the growth rate $\sigma$ reads, correct to $\mathcal{O}\left(\epsilon^2, \param \right)$:
\begin{align}
3\epsilon V^2 \sigma G &= \left(1 + 2\epsilon \psi\right)\left[\epsilon\dd{\psi}{X} \dd{Q}{X} + \left(1 + \epsilon \psi\right)\left(\dd{^2 Q}{X^2} -\param K^2 Q\right)\right] & &0 < X < 1,\label{E:Asymptotics:ODEwet}\\
0 &= Q & &1 < X < \frac{1}{V},\label{E:Asymptotics:ODEdry}\\
Q &= \dd{^4 G}{X^4} - 2\param K^2 \dd{^2 G}{X^2} +\param K^4 G & &0 < X<\frac{1}{V},\label{A:Asymptotics:state_BVP:pressure2shape}
\end{align}
with boundary conditions
\begin{align}
G &= 0 = \dd{G}{X} = \dd{Q}{X} & &\text{at}~X = 0,\label{E:Asymptotics:BC_at_0}\\
Q + \frac{\epsilon V}{20}\ddp{Q}{X} &= \epsilon\left(\dd{\psi}{X} + G + K^2\right)  & &\text{at}~X= 1,\label{E:Asymptotics:pressure_bc}\\
\dd{^2 G}{X^2} -\param \poisson K^2 G &= 0 = \dd{^3 G}{X^3} - (2-\poisson)\param K^2 \dd{G}{X} & &\text{at}~X = \frac{1}{V},\label{E:Asymptotics:BC_at_1}
\end{align}
\begin{align}\label{E:Asymptotics:jump_conds}
\left[G\right]_-^+= \left[\dd{G}{x}\right]_-^+ = \left[\dd{^2G}{X^2} + \frac{\epsilon V}{20}\dd{^3 G}{X^3}\right]_-^+&= 0, \\
\left[\dd{^3 G}{X^3} + \frac{\epsilon V}{20}\dd{^4 G}{X^4}\right]_-^+ &= 1 - \epsilon \psi(1).
\end{align}
Here the jump applies across the equilibrium meniscus position, located at $X = 1$. The growth rate $\sigma$ satisfies
\begin{equation}\label{E:Asymptotics:kinematic}
3V^2\sigma = -\left[1 + 2\epsilon \psi(1)\right]\left[\dd{Q}{X} + \frac{\epsilon V}{20}\dd{^2 Q}{X^2}\right]_{X = 1}.
\end{equation}
Note that the reduced problem~\eqref{E:Asymptotics:ODEwet}--\eqref{E:Asymptotics:kinematic} is independent of the sign of $\nu$ and $\aspect$, demonstrating that the growth rate $\sigma$ is independent of the wettability in the limit of small deformations.

We pose a bivariate asymptotic expansion of $G(X)$, $Q(X)$, and $\sigma$ in both $\epsilon$ and $\param$, using subscript $\left\{i,j\right\}$ to denote the term at $\mathcal{O}(\param^i \epsilon^j)$:
\begin{align}
G(X) &=    G_{0,0} + \param G_{1,0} + \epsilon G_{0,1} +  \param^2 G_{2,0} + \param \epsilon G_{1,1} + \epsilon^2 G_{0,2} + \dots,\label{E:Asymptotics:ExpansionG} \\
Q(X) &=  Q_{0,0} + \param Q_{1,0} + \epsilon Q_{0,1} +  \param^2 Q_{2,0} + \param \epsilon Q_{1,1} + \epsilon^2 Q_{0,2}+\dots,\label{E:Asymptotics:ExpansionQ} \\
\sigma(k) &= \sigma_{0,0} + \param \sigma_{1,0} + \epsilon \sigma_{0,1} +  \param^2 \sigma_{2,0} + \param \epsilon \sigma_{1,1} + \epsilon^2 \sigma_{0,2} + \dots. \label{E:Asymptotics:ExpansionSigma}
\end{align}

The particular hierarchy of problems arising from the asymptotic expansion~\eqref{E:Asymptotics:ExpansionG}--\eqref{E:Asymptotics:ExpansionSigma} depends on the relative sizes of $\epsilon$ and $\param$, but we must ensure that $V \gg |a|$ for lubrication theory to remain valid, and therefore
$\param  = \epsilon (V/|a|) \gg \epsilon$.  In Appendix~\ref{A:SmallDeformationAsymptotics}, we present the particular hierarchy of problems and solutions for the case $\epsilon \sim \param^2$, but this analysis can be readily transferred to other asymptotic balances which preserve $\delta \gg \epsilon$.

The leading order ($\mathcal{O}(1)$) and first order ($\mathcal{O}(\param)$) problems are independent of the relationship between $\epsilon$ and $\param$. We find that, in these problems, the pressure profile $Q_{i,0}(X),~i = 1,2$ is a linear function of $X$. However, to satisfy the no flux condition at $X =0$, this pressure perturbation is in fact constant, and thus from~\eqref{E:Asymptotics:kinematic} offers no contribution to $\sigma$, i.e.
\begin{equation}
\sigma_{0,0} = 0 = \sigma_{1,0}.
\end{equation}
We also find the leading order contribution to the perturbation to the channel shape to be
\begin{equation}\label{E:Asymptotics:ChannelShapeSolution}
G_{0,0} =\begin{cases}
\frac{X^2}{6}(X - 3) & 0 < X < 1,\\
\frac{1}{6}(1-X) & 1 < X < 1/V.
\end{cases}
\end{equation}
In particular, this means that $H(X=1) \sim -\nu V^3/3 + \order{\delta}$ as $|\nu| V^4 \to 0$, which agrees well with numerical solutions of the BVP~\eqref{E:LSA:ODEwet}--\eqref{E:LSA:kinematic} (figure~\ref{fig:CollapsedGrowthRates}a). This result confirms that the perturbation to the channel shape is negative (positive, respectively) for wetting (non-wetting) configurations, and that the leading order perturbation to the channel shape is independent of the wavenumber $k$, as suggested in \S\ref{S:LSA}. In addition, we find the Poisson's ratio $\poisson$ in the first order term, $G_{1,0}$, but not in the leading order term, $G_{0,0}$, demonstrating that the contribution from the dry region enters at lower order (the Poisson's ratio only enters the problem via the boundary conditions on the dry region, at $x = 1$): bending deformations are confined primarily to the wet region in the limit of small deformations, as expected.

\begin{figure}
\centering
\includegraphics[width = \textwidth]{figures/fig8_asymptotics.pdf}
\caption{(a) Numerically obtained values for the normalized perturbation to the channel width $H(x = x_0)$ and (b) normalized growth rate $\sigma$ as a function of the reduced wavenumber $k/k_c$. Each curve corresponds to a unique $(V,\nu)$ pair (the aspect ratio $a = 0.01$ is fixed), whose combination $\nu V^5 /a$ is indicated by the colors in the colorbar. The black dashed curves correspond to the asymptotic results~\eqref{E:Asymptotics:ChannelShapeSolution} and~\eqref{E:Asymptotics:FastestGrowingMode}, respectively. The numerical results are indistinguishable from the asymptotic curves for $\nu V^5 /a \lesssim 10^{-2}$. The inset in (b) is a semilogarithmic plot of the ratio between the numerically obtained values of the maximum growth rate $\sigma^*$ for $V = 0.1$, $0.2$, $0.3$, and $0.4$. (These curves terminate where the corresponding equilibria cease to exist, having violated the no contact condition~\eqref{E:Equilibria:NoTouchCond}.) The black dashed curve indicates the small deformation prediction $\sigma^* = \sigma^*_{SD}$ (equation~\eqref{E:Asymptotics:FastestGrowingMode}).}
\label{fig:CollapsedGrowthRates}
\end{figure}

Despite the dependence of the hierarchy of problems on the relationship between $\epsilon$ and $\param$, the first non-zero term in the expansion of $\sigma$~\eqref{E:Asymptotics:ExpansionSigma} is
\begin{equation}\label{E:Asymptotics:LeadingOrderSigma}
\sigma_{1,1} = \frac{K^2\left(1-2K^2\right)}{6V^2},
\end{equation}
regardless of the relationship between $\epsilon$ and $\param$. The first non-zero term in the expansion~\eqref{E:Asymptotics:ExpansionQ} is $Q_{0,1}$, because the terms corresponding to the destabilizing transverse curvature, and stabilizing in-plane curvature contributions to the pressure enter at this order, but we find that $Q_{0,1}$ is constant; the first term in~\eqref{E:Asymptotics:ExpansionQ} with a non-zero gradient, which sets the growth rate, comes in at the next order, $\mathcal{O}(\epsilon \param)$.

Noting that $\sigma_c = \epsilon\param/V^2$, substituting~\eqref{E:Asymptotics:LeadingOrderSigma} in to the expansion~\eqref{E:Asymptotics:ExpansionSigma}, gives
\begin{equation}\label{E:Asymptotics:SigmaResult}
\frac{\sigma(K)}{\sigma_c} =  \frac{K^2\left(1-2K^2 \right)}{6}  + \mathcal{O}(\param).
\end{equation}
Equation~\eqref{E:Asymptotics:SigmaResult} gives the envelope of the curves $\sigma(k)$ (figure~\ref{fig:CollapsedGrowthRates}b). Moreover, and as expected from~\eqref{E:Asymptotics:ExpansionSigma}, numerical solutions with larger values of $\param = \nu V^5 /|a|$ deviate more significantly from this envelope.

By maximizing~\eqref{E:Asymptotics:SigmaResult} with respect to $K$, we find that the small deformation estimates of the fastest growing mode, denoted $k^*_{SD}$, and the corresponding growth rate, denoted $\sigma^*_{SD}$, are
\begin{equation}\label{E:Asymptotics:FastestGrowingMode}
\sigma^*_{SD}= \frac{1}{48}\sigma_c = \frac{1}{48}\frac{\nu^2 V^7}{|\aspect|}, \qquad k^*_{SD}= \frac{1}{2}k_c = \frac{1}{2}\sqrt{\frac{\nu V^3}{\aspect}}.
\end{equation}
which agree well with numerical solutions of the BVP (see inset of figure~\ref{fig:CollapsedGrowthRates}b, in which perfect agreement would correspond to a constant value of one in the abscissa). Numerical solutions with larger values of $V$, and thus larger values of $\epsilon = |\nu| V^4$ for a given $\nu$, `peel off' from the asymptotic prediction~\eqref{E:Asymptotics:FastestGrowingMode} at a lower value of $\delta$, as we would expect. The asymptotic result~\eqref{E:Asymptotics:FastestGrowingMode} overestimates the fastest growing mode as $\param$ grows from zero because the increased penalizing effect of in-plane bending, which suppressed the growth rate relative to the asymptotic result, is stronger than the effect of increased deformation of the base state.

\section{Summary}\label{S:Conclusion}
%very broad description
In this paper, we set out to understand the periodic pattern that is observed in experiments in which liquid condenses into deformable microchannels. In doing so, we identified a novel instability mechanism that is driven by a competition between interfacial curvatures at the liquid surface, and is mediated by the elasticity of the channel in response to liquid pressure. This bendocapillary instability is theoretically possible in the same channel for both wetting and non-wetting liquids, unlike the analogous instability in rigid, tapered channels~\citep{AlHousseiny2012NaturePhysics}.

%what did we do: studied system which may be susceptible, scaling argument, develop mathematical model, linear stability analysis, asymptotics (and why did we do these
We studied this mechanism in detail by considering a simple system that may be susceptible to the instability. We developed a mathematical model of this system, which was simplified by exploiting the small aspect ratio of the both the channel walls and cavity between them, allowing us to appeal to linear plate theory to describe solid deformations and lubrication theory to describe the flow. In non-dimensionalizing this system, we identified three dimensionless parameters, relating to the ability of the liquid to deform the channel walls, the liquid volume, and the channel aspect ratio. Equilibrium configurations, which form the base states of the system, are parametrized by the first two of these. 

The rest of the paper focused on the linear stability of these equilibria to in-plane perturbations. To assess this stability, we set out the linearized equation that must be satisfied by perturbations; these equations illustrate explicitly the two ways that the elastic case differs from the tapered, rigid case: the bulk channel deformation is set by the liquid pressure, and the perturbation induces an additional elastic response of the channel, altering its geometry. Numerical solutions of the linearized equations suggested three key results, which were verified analytically in the limit of small deformations: (1) both wetting and non-wetting equilibria are always unstable to perturbations of a sufficiently small wavenumber, (2) the growth rate of the fastest growing mode is highly sensitive to the amount of liquid within the channel ($\sigma \sim V^7$), and (3) the additional elastic response to the perturbation always enhances the destabilizing in-plane curvature contribution, and thus tends to promote instability.

%% above paragraph replaced below %%
%By performing a linear stability analysis of its equilibrium configurations, we identified that the system is unstable to perturbations of a sufficiently small wavenumber. The linearized equations elucidate the two main ways that the elastic case differs from the tapered, rigid case in two important ways: the bulk channel elasticity is set by the liquid pressure, and the channel responds to the perturbation in a way that tends to enhance the difference between the in-plane and transverse curvatures, increasing the range of unstable wavenumbers. The growth rate of the fastest growing mode is highly sensitive to the amount of liquid within the channel (parametrized by the cross-sectional volume $V$); in particular, we identified that the growth rate $\sigma \sim V^7$ in the case of small deformations and negligible in-plane bending from both a simple scaling argument and a formal asymptotic analysis.

%but we only looked at constant volume cases, introduce changing volume 
In this first part one of two, we considered only the no condensation case, in which equilibria have a constant cross sectional volume $V$; in the second part, we describe how a non-zero condensation rate (i.e. how changing the volume of liquid in the channel) changes the picture presented here. The sensitive dependence on the amount of liquid in the channel suggests that the results may of this paper may be significantly different when a non-zero condensation rate is included. In the motivating experiments, however, condensation is very slow and an order of magnitude estimate for the experimentally observed wavelength may therefore be obtained from the quasistatic analysis presented here. Using values from~\cite{Seemann2011JPhysCondMat}, we find that $\nu \approx -12$ and $\aspect \approx -0.38$ in these experiments; the wavelength of approximately $200~\si{\micro \meter}$ that is observed experimentally (see figure~\ref{fig:Experiments}) agrees in its order of magnitude with the asymptotic result~\eqref{E:Asymptotics:FastestGrowingMode}, which predicts a wavelength of $370~\si{\micro \meter}$ when the channel is half full, $V = 0.5$. Smaller values of $V$ result in predictions of longer wavelengths, but these are of the same order of magnitude (e.g. with $V = 0.1$, the scaling argument predicts a wavelength of $4~\si{\milli \meter}$.) 

%our results suggests that configurations are always unstable to \emph{some} wavelength, but we don't expect to see instability in practice because of (and not limited to) (a) slow growth rates (b) finite size effects and (c) contact angle hysteresis
As mentioned, our results suggests that narrow, deformable channels on small scales are always unstable to perturbations of sufficiently long wavelengths. There are several reasons why we do not expect this to be the case in practice. Firstly, realistic channels will have a finite extent in the $y$-direction (as it is referred to here); the maximum wavelength of perturbations will be restricted to this maximum length. Secondly, configurations with stiff walls (small $\nu$), will have fastest growing modes whose growth rates are very small (the growth rate $\sigma \sim \nu^2$ see equation~\eqref{E:Asymptotics:FastestGrowingMode}), allowing processes that occur on longer timescales (e.g. evaporation or condensation) to interact with the growth of the bendocapillary instability. Finally, we made a series of assumptions on the physical processes included -- and neglected -- in our model. Perhaps most notably, we have neglected dynamic contact angle effects, shown to be important in controlling the dynamics in similar elastocapillary systems~\citep{Bradley2021}. Contact angle hysteresis is expected to reduce the growth rate of perturbations: protrusions of a perturbation, which are advancing interfaces, will have a higher contact angle, and thus smaller magnitude pressure, than invaginations, which are receding interfaces~\citep{Bradley2021}.


%implications: can this instability be exploited
We postulate that the fact this bendocapillary instability has not been described in detail previously might suggest is has not been encountered in any situations in which it is a hindrance. We hope, therefore, that the insights offered in this paper might motivate further will motivate further experimental studies to quantify and understand the interface dynamics in bendocapillary systems and identify situations in which it may be exploited. 



\appendix
\section{Asymptotic Analysis for $\epsilon \sim \delta^2$}\label{A:SmallDeformationAsymptotics}
In this appendix, we set out, and present an analysis of, the hierarchy of equations up to $\order{\delta}$, that arise from inserting the asymptotic expansion~\eqref{E:Asymptotics:ExpansionG}--\eqref{E:Asymptotics:ExpansionSigma} into the system of equations~\eqref{E:Asymptotics:ODEwet}--\eqref{E:Asymptotics:kinematic} in the case that $\epsilon \sim \delta^2$. To reflect this asymptotic order, we introduce $\beta = \epsilon / \delta^2 = \order{1}$ here.

The leading order ($\order{1}$) problem, which is independent of the balance between $\epsilon$ and $\delta$ reads
\begin{align}
0&= \dd{^2Q_0}{X^2} & &0 < X<1,\label{A:Asymptotics:0thOrder:ODEwet}\\
0&= Q_0  & &1 < X< \frac{1}{V},\\
Q_0 &= \dd{^4 G_0}{X^4} & & 0 < X < \frac{1}{V},
\end{align}
\begin{align}
G_0 &= 0 = \dd{G_0}{X}= \dd{Q_0}{X} & &\text{at}~X = 0,\label{A:Asymptotics:0thOrder:x0bc}\\
Q_0 &=0 & &\text{at}~X = 1,\label{A:Asymptotics:0thOrder:pressurebc}\\
\dd{^2 G_0}{X^2} &=0 = \dd{^3 G_0}{X^3} & &\text{at}~X = \frac{1}{V},
\end{align}
\begin{equation}
\left[G_0\right]_-^+ =0 = \left[\dd{G_0}{X}\right]_-^+ = \left[\dd{^2 G_0}{X^2}\right]_-^+, \quad  \left[\dd{^3 G_0}{X^3}\right]_-^+ = 1,
\end{equation}
\begin{equation}\label{A:Asymptotics:0thOrder:kinematic}
3V^2 \sigma_0 = -\left.\dd{Q_0}{X}\right|_{X=1}.
\end{equation}
From~\eqref{A:Asymptotics:0thOrder:ODEwet},  the pressure $Q_0$ is a linear function of $X$. However, from~\eqref{A:Asymptotics:0thOrder:x0bc}--\eqref{A:Asymptotics:0thOrder:pressurebc}, this linear function has no slope and passes through zero; we therefore have $Q_0 = 0$, and from~\eqref{A:Asymptotics:0thOrder:kinematic}, $\sigma_0 = 0$. The solution to~\eqref{A:Asymptotics:0thOrder:ODEwet}--\eqref{A:Asymptotics:0thOrder:kinematic} for the channel shape is
\begin{equation}\label{A:Asymptotics:0thOrder:solution}
G_0 = \begin{cases}
-\frac{X^2}{6}\left(3-X\right) & 0 < X < 1,\\
-\frac{1}{6}\left(X-1\right) & 1< X < 1/V.
\end{cases}
\end{equation}

The first order ($\order{\delta}$) problem is similar, and is also independent of the relative sizes of $\epsilon$ and $\delta$. Again, we get no pressure contribution, $Q_1 = 0$ (the equations for $Q_1$ are identical to those for $Q_0$) and thus $\sigma_1 = 0$. The shape contribution $G_1$ is non-trivial, but is not required for the determination of the leading order behaviour for $\sigma$, and we therefore do not state it here. We note, however, that the Poisson's ratio $\poisson$ first appears in this term, highlighting the lower order contribution of the dry regions.

%%%%%%%%%%%%%%%%%%%%%%%%%
The  $\mathcal{O}(\delta^2)$ problem may be expressed simply be exploiting the  $\mathcal{O}(1)$ and $\mathcal{O}(\delta)$ problems. We give only this simplified form here:
\begin{align}
 \dd{^2 Q_2}{X^2}&=0 & &0 < X<1,\\
Q_2 &= 0 & &1 < X< \frac{1}{V},\\
Q_2 &= \dd{^4 G_2}{X^4} - 2K^2\dd{G_1}{X^2}+ K^4 G_0 & & 0 < X < \frac{1}{V},
\end{align}
\begin{align}
G_2 &= 0 = \dd{G_2}{X}= \dd{Q_2}{X} & &\text{at}~X = 0,\\
Q_2 &=G_0+ K^2+ \dd{\psi}{X} & &\text{at}~X = 1,\label{A:Asymptotics:2ndOrder:PressureBC}\\
\dd{^2 G_2}{X^2} - \poisson K^2 G_1 &=0 = \dd{^3 G_0}{X^3}  - (2- \poisson)K^2 \dd{G_1}{X}& &\text{at}~X = \frac{1}{V},
\end{align}
\begin{equation}
\left[G_2\right]_-^+ =0 = \left[\dd{G_2}{X}\right]_-^+ = \left[\dd{^2 G_2}{X^2} + \frac{\beta V}{20}\dd{^3G_0}{X^3}\right]_-^+ = \left[\dd{^3 G_2}{X^3} + \frac{\beta V}{20}\dd{^4 G_0}{X^4}\right]_-^+,
\end{equation}
\begin{equation}
3V^2 \sigma_2 = -\left.\dd{Q_2}{X}\right|_{X=1}.
\end{equation}
Crucially, the boundary condition~\eqref{A:Asymptotics:2ndOrder:PressureBC} is inhomogeneous, in contrast to the corresponding boundary condition for the lower order problems. We therefore find the first non-zero pressure term in the expansion~\eqref{E:Asymptotics:ExpansionQ} for the channel shape perturbation to be
\begin{equation}\label{A:Asymptotics:2ndOrder:solQ2}
Q_2 = \begin{cases}
K^2 - \frac{1}{2} & 0 <X  < 1,\\
0 & 1 < X < 1/V,
\end{cases}
\end{equation}
where we have used $G_0$, from~\eqref{A:Asymptotics:0thOrder:solution}, to obtain $Q_2$. (Note that for a relationship other than $\epsilon\sim \param^2$, the boundary condition~\eqref{A:Asymptotics:2ndOrder:PressureBC} would be unchanged because the shape contributions at higher order all take the same value at $ X= 1$.) This leading order pressure contribution is constant in the liquid and thus again offers no contribution to the growth rate, hence $\sigma_2 = 0$.

To obtain a non-zero term in the expansion for $\sigma$ we must proceed to $\mathcal{O}(\param^3)$, where we find that
\begin{align}
 \dd{^2 Q_3}{X^2} - K^2 Q_2 &= 0 & &0 < X<1,\label{A:Asymptotics:3ndOrder:odewet}\\
Q_3 &= 0 & &1 < X< \frac{1}{V},\\
Q_3 &= \dd{^4 G_3}{X^4} - 2K^2\dd{G_2}{X^2}+ K^4 G_1 & & 0 < X < \frac{1}{V},
\end{align}
\begin{align}
G_3 &= 0 = \dd{G_3}{X}= \dd{Q_3}{X} & &\text{at}~X = 0,\label{A:Asymptotics:3ndOrder:x0bc}\\
Q_3 &=\beta G_1 & &\text{at}~X = 1,\label{A:Asymptotics:3ndOrder:PressureBC}\\
\dd{^2 G_2}{X^2} - \poisson K^2 G_1 &=0 = \dd{^3 G_0}{X^3}  - (2- \poisson)K^2 \dd{G_1}{X}& &\text{at}~X = \frac{1}{V},
\end{align}
\begin{equation}
\left[G_3\right]_-^+ =0 = \left[\dd{G_3}{X}\right]_-^+ = \left[\dd{^2 G_3}{X^2} + \frac{\beta V}{20}\dd{^3G_1}{X^3}\right]_-^+ = \left[\dd{^3 G_3}{X^3} + \frac{\beta V}{20}\dd{^4 G_1}{X^4}\right]_-^+,
\end{equation}
\begin{equation}\label{A:Asymptotics:3ndOrder:kinematic}
3V^2 \sigma_3 = -\left.\dd{Q_3}{X}\right|_{X=1}.
\end{equation}
From~\eqref{A:Asymptotics:3ndOrder:odewet} and~\eqref{A:Asymptotics:3ndOrder:x0bc}, we find that
\begin{equation}\label{A:Asymptotics:3ndOrder:solutionQ3}
\dd{Q_3}{X} = K^2 Q_2 X \qquad 0 < X < 1.
\end{equation}
Inserting~\eqref{A:Asymptotics:3ndOrder:solutionQ3} into~\eqref{A:Asymptotics:3ndOrder:kinematic}, and using~\eqref{A:Asymptotics:2ndOrder:solQ2} gives
\begin{equation}\label{A:Asymptotics:3ndOrder:solutionsigma3}
\sigma_3  = -\frac{K^2}{6V^2}\left(2K^2 - 1\right).
\end{equation}
This is the leading order term in the expansion~\eqref{E:Asymptotics:ExpansionSigma} for $\sigma$.

Undoing the various variable changes introduced in \S\ref{S:Asymptotics}, the result~\eqref{A:Asymptotics:3ndOrder:solutionsigma3} gives
\begin{equation}\label{A:Asymptotics:ThirdOrder:SigmaSolution}
\sigma\sim \frac{\nu^2 V^7}{a}F\left(\frac{k}{k_c}\right)
\end{equation}
where
\begin{equation}\label{A:Asymptotics:SigmaSolutionParticulars}
F(\xi) = -\frac{\xi^2}{6}\left(2\xi^2 -1\right), \qquad k_c = \left(\frac{\nu V^3}{a}\right)^{1/2}.
\end{equation}


\bibliographystyle{jfm}
% Note the spaces between the initials
\bibliography{mybib.bib}

\end{document}
